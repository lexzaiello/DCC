\documentclass{beamer}
\usepackage{minted}

\newcommand{\pie}[1]{\Pi_{#1}}
\newcommand{\type}[1]{\text{Type } #1}
\newcommand{\turnstile}{\vdash}
\newcommand{\sigl}[1]{\(\texttt{::[}#1\texttt{]}\)}
\newcommand{\sigle}[1]{\texttt{::[}#1\texttt{]}}

\usetheme{Madrid}

\title{The Dependent Combinator Calculus}
\author{Alexandra Aiello}
\date{\today}

\begin{document}

\frame{\titlepage}

\begin{frame}{Overview}
  \begin{itemize}
    \item The \textit{DCC} achieves dependent typing without variables by interpreting sigma types (i.e., \(s := (fst : \alpha) \times (snd : \beta x)\) in Lean) as a future function application.
    \item We make use of a correspondence between \(snd\) projection and variable substitution to emulate variables under \(\Pi\) binders
    \item This correspondence also acts as a bridge between curried functions and \(n\)-argument functions, like combinators.
  \end{itemize}
\end{frame}

\begin{frame}{AST}



\end{frame}

\begin{frame}{Evaluation Rules}
  \begin{itemize}
    \item 
  \end{itemize}
\end{frame}

\begin{frame}{Inference Rule: Function Application}

  \begin{itemize}
    
  \end{itemize}
  
\end{frame}

\end{document}

