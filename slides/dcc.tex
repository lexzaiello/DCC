\documentclass{beamer}
\usepackage{fontspec}
\usepackage{minted}
\usepackage[
  backend=biber,
  style=alphabetic,
  sorting=ynt
]{biblatex}
\addbibresource{./refs.bib}

\newcommand{\type}[1]{\text{Type } #1}
\newenvironment{bitemize}[1]{\begin{itemize}\setlength{\itemsep}{1em}#1\end{itemize}}

\AtBeginSection[]{
  \begin{frame}
    \vfill
    \centering
    \begin{beamercolorbox}[sep=8pt,center,shadow=true,rounded=true]{title}
      \usebeamerfont{title}\insertsectionhead\par%
    \end{beamercolorbox}
    \vfill
  \end{frame}
}

\setmainfont{DejaVu Serif} 
\setsansfont{DejaVu Sans}
\setmonofont{JuliaMono}[
    Scale=0.9,
    Extension = .ttf,
    UprightFont = *-Regular,
    BoldFont = *-Bold,
    ItalicFont = *-RegularItalic,
    BoldItalicFont = *-BoldItalic
]

\usemintedstyle{tango}

\newcommand{\leane}[1]{\mintinline{lean4}|#1|}
\newcommand{\pie}[1]{\Pi_{#1}}
\newcommand{\type}[1]{\text{Type } #1}
\newcommand{\turnstile}{\vdash}
\newcommand{\sigl}[1]{\(\texttt{::[}#1\texttt{]}\)}
\newcommand{\sigle}[1]{\texttt{::[}#1\texttt{]}}

\usetheme{Madrid}

\title{The Dependent Combinator Calculus}
\author{Alexandra Aiello}
\date{\today}

\begin{document}

\frame{\titlepage}

\begin{frame}{Problem Statement: The \(\Pi\)-Combinator Mismatch}
  \begin{itemize}
    \item Lambdas are curried \(\lambda x.\lambda y.M\), and so are their types \(\forall (x : \alpha), \forall (y : \beta), \ldots\).
    \item Combinators are uncurried (\(K(x, y) = x\). Fixed number of known arguments.
    \item Past approaches treated combinator types as uncurried.
      \begin{itemize}
        \item Altenkirch and I used \(\Pi\) combinators, or meta combinators. Many disadvantages.
      \end{itemize}
    \item By uncurrying combiantor types, dependent types become simple, and elegant.
  \end{itemize}    
\end{frame}

\begin{frame}[fragile]{Central Thesis: Combinator-Sigma Correspondence}
  \begin{figure}
\begin{minted}[mathescape, fontsize=\small, linenos]{lean4}
/-- Interpret a function on `Σ x : α, β x` as a dependent function with two arguments.
This also exists as an `Equiv` as `Equiv.piCurry γ`. -/
def Sigma.curry {γ : ∀ a, β a → Type*} (f : ∀ x : Sigma β, γ x.1 x.2) (x : α) (y : β x) : γ x y :=
  f ⟨x, y⟩

/-- Interpret a dependent function with two arguments as a function on `Σ x : α, β x`.
This also exists as an `Equiv` as `(Equiv.piCurry γ).symm`. -/
def Sigma.uncurry {γ : ∀ a, β a → Type*} (f : ∀ (x) (y : β x), γ x y) (x : Sigma β) : γ x.1 x.2 :=
  f x.1 x.2

-- in Mathlib Equiv module
def piCurry {α} {β : α → Type*} (γ : ∀ a, β a → Type*) :
    (∀ x : Σ i, β i, γ x.1 x.2) ≃ ∀ a b, γ a b where
  toFun := Sigma.curry
  invFun := Sigma.uncurry
  left_inv := Sigma.uncurry_curry
  right_inv := Sigma.curry_uncurry
\end{minted}
    \caption{Equivalence between dependent functions and sigmas in Mathlib (curry-uncurry). Sigmas carry the variable and substitution as data, whereas \(\forall\) substitution is not data.}
  \end{figure}
\end{frame}

\begin{frame}{Central Thesis: Combinator-Sigma Correspondence}
  A function on a sigma term corresponds to a dependent combinator equation. A sigma type corresponds to the uncurried type of a dependent combinator. E.g., the dependent \(K\) combinator:
  \begin{equation}
    K((\alpha : \type{m}), (\beta : \alpha \rightarrow \type{n}), (x : \alpha), (y : \beta x)) = x
  \end{equation}
\begin{minted}[mathescape, fontsize=\small, linenos]{lean4}
-- #check Sigma.mk
Sigma.mk.{u, v} {α : Type u} {β : α → Type v} (fst : α) (snd : β fst) : Sigma β

-- #check Sigma.fst. This is = x
Sigma.fst.{u, v} {α : Type u} {β : α → Type v} (self : Sigma β) : α
  \end{minted}
\end{frame}

\begin{frame}{Central Thesis: Combinator-Sigma Correspondence}
  \begin{itemize}
    A sigma term \leane{Sigma.mk (x : α) (y : ∀ (x : α), β x)} corresponds to a dependent combinator equation \leane{C(x, y) = M}. A sigma type \leane{
  \end{itemize}    
\end{frame}

\end{document}
