\documentclass{beamer}
\usepackage{fontspec}
\usepackage{minted}
\usepackage[
  backend=biber,
  style=alphabetic,
  sorting=ynt
]{biblatex}
\addbibresource{./refs.bib}

\newenvironment{bitemize}[1]{\begin{itemize}\setlength{\itemsep}{1em}#1\end{itemize}}

\AtBeginSection[]{
  \begin{frame}
    \vfill
    \centering
    \begin{beamercolorbox}[sep=8pt,center,shadow=true,rounded=true]{title}
      \usebeamerfont{title}\insertsectionhead\par%
    \end{beamercolorbox}
    \vfill
  \end{frame}
}

\setmainfont{DejaVu Serif} 
\setsansfont{DejaVu Sans}
\setmonofont{JuliaMono}[
    Scale=0.9,
    Extension = .ttf,
    UprightFont = *-Regular,
    BoldFont = *-Bold,
    ItalicFont = *-RegularItalic,
    BoldItalicFont = *-BoldItalic
]

\usemintedstyle{tango}

\newcommand{\leane}[1]{\mintinline{lean4}|#1|}
\newcommand{\pie}[1]{\Pi_{#1}}
\newcommand{\type}[1]{\text{Type } #1}
\newcommand{\turnstile}{\vdash}
\newcommand{\sigl}[1]{\(\texttt{::[}#1\texttt{]}\)}
\newcommand{\sigle}[1]{\texttt{::[}#1\texttt{]}}

\usetheme{Madrid}

\title{The Dependent Combinator Calculus}
\author{Alexandra Aiello}
\date{\today}

\begin{document}

\frame{\titlepage}

\begin{frame}{Overview}
  \begin{itemize}
    \item The \textit{DCC} achieves dependent typing without variables by interpreting sigma types (i.e., \leane{(fst : \alpha) \times (snd : \beta x)} in Lean) as a future function application.
    \item We make use of a correspondence between \(Sigma.snd\) projection and variable substitution to emulate \(\Pi\) binders
    \item This correspondence also acts as a bridge between curried functions and \(n\)-argument functions, like combinators by allowing arguments to be treated as data to be applied later.
  \end{itemize}
\end{frame}

\begin{frame}{Sigma-Curry Correspondence}
  \begin{itemize}
    \item Combinator evaluation are usually written with all arguments (e.g., \(K(x, y) = x\))
    \item Writing combinators in point-free / ``curried'' notation feels unnatural and results in complex types
    \item Past approaches, Altenkirch's \cite{altenkirch} used helper combinators form types.
      \begin{itemize}
        \item Downside: the kernel can't tell if two types with the same shape are equal
        \item Downside: AST becomes very large---meta combinators also need types
      \end{itemize}
    \item Sigma types store expressions as data, and emulate \(\Pi\)-like substitution in a universal way.
      \begin{itemize}
        \item No meta combiantor hierarchy
        \item No AST explosion
      \end{itemize}
  \end{itemize}
\end{frame}

\begin{frame}{Sigma Types}
  \begin{itemize}
    \item Written as \mintinline{lean4}|::[a, b]|. Expands to \mintinline{lean4}|.app (.app .cons a) b|
    \item Notation: \mintinline{lean4}|::[a, b, c, ...] = ::[a, ::[b, ::[c, ...]]]|
    \item Reduction rule: \mintinline{lean4}|| (.app ::[x, f], fn) => pure (.app fn, f, x)|
      \begin{itemize}
        \item \(fn\) above projects over the sigma. \(fn = id\) induces
      \end{itemize}
  \end{itemize}  
\end{frame}

\section{AST}

\begin{frame}[fragile]{AST}
  \begin{minted}[mathescape, fontsize=\small, linenos]{lean4}
abbrev Level := ℕ

inductive Expr where
  | ty     : Level → Expr
   -- Cons can be partially applied, so it is a combinator
  | cons   : Expr
  | app    : Expr → Expr → Expr
   -- Nil "downgrades" a term to a type
   -- nil (α : Ty m) : α → Ty m
   -- useful for typing arguments like ∀ (α : Type), ...
  | nil    : Level → Expr
  -- The I combinator in SK
  | id     : Level → Expr
  -- Dependent K combinator
  | const  : Level → Level → Expr
  -- Nondependent K. Closes the loop of type dependency.
  | const' : Level → Level → Expr
\end{minted}

Cont \ldots
\end{frame}

\begin{frame}[fragile]{AST (cont \ldots)}
  \begin{minted}[mathescape, fontsize=\small, linenos]{lean4}
  inductive Expr where
  ...
  /- form a new sigma pair from (both f g x) ::[(f x), (g x)]
     but does not force application of (f x) (g x)
     combined with snd projection results in S -/
  | both   : Level → Level → Level → Expr
  -- Branch on def-eq. For practical applications.
  | eq     : Level → Level → Level → Expr
  \end{minted}
\end{frame}

\begin{frame}[fragile]{Evaluation Rules}
  \begin{minted}[mathescape, fontsize=\small, linenos]{lean4}
inductive is_step : Expr → Expr → Prop
  | sapp   : is_step ($ ::[x, f], fn) ($ fn, f, x)
  | nil    : is_step ($ (nil _o), α, _x) α
  | id     : is_step ($ (.id _o), _α, x) x
  | const  : is_step ($ (.const _o _p), _α, _β, c, _x) c
  | const' : is_step ($ (.const' _o _p), _α, _β, c, _x) c
  | both   : is_step ($ (.both _o _p _q), _α, _β, _γ, f, g, x)
    ::[($ f, x), ($ g, x)]
  | eq_yes : a == b → is_step
    ($ (.eq _o _p), _α, _β, fn_yes, fn_no, a, b)
    (.app fn_yes a)
  | eq_no  : a ≠ b → is_step
    ($ (.eq _o _p), _α, _β, fn_yes, fn_no, a, b)
    (.app fn_no b)
  | left   : is_step f f' → is_step ($ f, x) ($ f', x)
  | right  : is_step x x' → is_step ($ f, x) ($ f, x')
  \end{minted}
\end{frame}

%\section{Inference Rules}
%
%\begin{frame}{Notation}
%  
%
%\end{frame}
%
%\begin{frame}{Inference Rule: Application}
%  \begin{equation}
%    
%  \end{equation}
%\end{frame}

%\begin{frame}{Evaluation Rules}
%  \begin{itemize}
%    \item 
%  \end{itemize}
%\end{frame}

%\begin{frame}{Inference Rule: Function Application}
%
%  \begin{itemize}
%    
%  \end{itemize}
%  
%\end{frame}

\begin{frame}[allowframebreaks]{References}
    \printbibliography
\end{frame}

\end{document}

