%  Copyright (C) 2025, 2026  Dowland Aiello
%
%  This program is free software: you can redistribute it and/or modify
%  it under the terms of the GNU General Public License as published by
%  the Free Software Foundation, either version 3 of the License, or
%  (at your option) any later version.
%
%  This program is distributed in the hope that it will be useful,
%  but WITHOUT ANY WARRANTY; without even the implied warranty of
%  MERCHANTABILITY or FITNESS FOR A PARTICULAR PURPOSE.  See the
%  GNU General Public License for more details.
%
% You should have received a copy of the GNU General Public License
% along with this program.  If not, see <https://www.gnu.org/licenses/>.

\documentclass[12pt]{article}

\usepackage{fvextra}
\fvset{breaklines=true}
\usepackage{tabularx}
\usepackage{needspace}
\usepackage{amsmath}
\usepackage{amsthm}
\usepackage{amssymb}
\usepackage[T1]{fontenc}
\usepackage{float}
\usepackage{scrextend}
\usepackage{fancyhdr}
\pagestyle{fancy}
\fancyfoot[L]{\footnotesize \textcopyright \ 2025 Dowland Aiello}
\fancyfoot[R]{\footnotesize Licensed under GNU GPL v3}
\newcommand{\inter}{\cap}
\newcommand{\iunion}{\bigcup}
\newcommand{\iintersect}{\bigcap}
\newcommand{\union}{\cup}
\newcommand{\diff}{\setminus}
\newcommand{\powerset}[1]{\(\mathcal{P}(#1)\)}
\newcommand{\epowerset}[1]{\mathcal{P}(#1)}
\newcommand{\power}[1]{\(\mathcal{P}(#1)\)}
\newcommand{\epower}[1]{\mathcal{P}(#1)}
\newcommand{\Letsets}[1]{Let #1 be sets}
\newcommand{\Letdef}[2]{Let \(#1 = #2\)}
\newcommand{\esigma}[2]{\displaystyle \sum_{#1}^{#2}}
\newcommand{\eprod}[1]{\displaystyle\prod_{#1}{}}
\newcommand{\Sup}[1]{Suppose #1}
\newcommand{\esup}[1]{suppose #1}
\newcommand{\bycontra}[1]{for the sake of contradiction that #1}
\newcommand{\Conclude}[1]{\textbf{Therefore, #1}}
\newcommand{\eby}[1]{\textit{by #1}}
\newcommand{\By}[1]{By \textit{#1}}
\newenvironment{case}[1]{\textbf{#1 case:}\\\begin{addmargin}[1cm]{0pt}}{\\\end{addmargin}}
\newenvironment{ecase}[1]{\textit{#1}\\\begin{addmargin}[1cm]{0pt}}{\\\end{addmargin}}
\newenvironment{inductive}{\begin{case}{Inductive}}{\end{case}}
\newenvironment{base}{\begin{case}{Base}}{\end{case}}
\newcommand{\mpiff}[1]{\textbf{(\(\Rightarrow\))} #1}
\newcommand{\mpriff}[1]{\textbf{(\(\Leftarrow\))} #1}
\newcommand{\Byinduction}[1]{We proceed by induction on #1}
\newcommand{\integers}[1]{\(\{#1\} \subseteq \mathbb{Z}\)}
\newcommand{\nats}[1]{\(\{#1\} \subseteq \mathbb{N}\)}
\newcommand{\integerspos}[1]{\(\{#1\} \subseteq \mathbb{Z}^{+}\)}
\newcommand{\integersneg}[1]{\(\{#1\} \subseteq \mathbb{Z}^{-}\)}
\newcommand{\integer}[1]{\(#1 \in \mathbb{Z}\)}
\newcommand{\integerpos}[1]{\(#1 \in \mathbb{Z}^+\)}
\newcommand{\integerneg}[1]{\(#1 \in \mathbb{Z}^-\)}
\newcommand{\einteger}[1]{#1 \in \mathbb{Z}}
\newcommand{\nat}[1]{\(#1 \in \mathbb{N}\)}
\newcommand{\enat}[1]{#1 \in \mathbb{N}}
\newcommand{\setwith}[3]{\(#1 = \{#2 : #3\}\)}
\newtheorem{theorem}{Theorem}
\newtheorem{lemma}{Lemma}
\newcommand{\pie}[1]{\Pi_{#1}}
\newcommand{\type}[1]{\text{Type } #1}
\newcommand{\turnstile}{\vdash}
\newcommand{\sigl}[1]{\(\texttt{::[}#1\texttt{]}\)}

\usepackage[margin=0.75in]{geometry}
\usepackage{fitch}
\usepackage{bbm}
\usepackage{stmaryrd}
\usepackage{hyperref}
\usepackage{mathrsfs}
\usepackage{mathtools}
\usepackage{rsfso}
\usepackage{graphicx}
\usepackage[
  backend=biber,
  style=alphabetic,
  sorting=ynt
]{biblatex}
\addbibresource{./refs.bib}


\author{Alexandra Aiello}
\date{December 19, 2025}
\title{The Dependent Arithmetic Machine}

\begin{document}

This is the type of a sigma \sigl{x, f}. Pairs are created with cons (\texttt{::}).
I use the notation \sigl{x_{1}, \cdots x_{n}} to denote \sigl{x_{1}, \text{\sigl{x_{2}, \cdots}}}.
Sigmas can be nested, and cons can be partially applied.
However, this means that type-checking cons is a special case, and doesn't follow directly from the general application rule.
I think this is reasonable, since it is so primitive.

Sigmas internalize the notion of projection by accepting a ``projector'' (\(\pi\)) function, which corresponds to function application where \(\pi = \text{id}\).
As such, there are no dedicated \texttt{fst} or \texttt{snd} combinators.

\begin{equation}
  \frac{
    \Gamma \vdash, (\alpha : \type{m}), (\beta : \type{n}), (x : \alpha), (y : \beta)
  }{
    \sigl{\alpha, \beta} : 
  }
\end{equation}

These are the types of the core DCC combinators.

Note that I am denoting types like \(\forall (x : \alpha), \beta(x)\) in point-free notation as \(\beta\), since
there are no \(\Pi\) binders in the DCC. This is explained more in the application inference rule.

I am using standard \(\Pi\) and \(\times\) notation here for legibility, and since I have not yet completed all the DCC tyeps in the latest, most sound version. Later in this overview, I outline the challenges I encountered, and the progress I have made towards finishing them.

\begin{equation}
\frac{}{
  \text{id}(m) : \Pi (\alpha : \type{m}) (x : \alpha), \alpha
}\\
\end{equation}

\begin{equation}
\frac{}{
  \text{nil}(m) : \Pi (\alpha : \type{m}) (x : \alpha), \type{m}
}\\
\end{equation}

\begin{equation}
\frac{}{
  \text{const'}(m,  n) : \Pi (\alpha : \type{m}) (\beta : \type{n}) (x : \alpha) (y : \beta), \alpha
}\\
\end{equation}

\begin{equation}
\frac{}{
  \text{const}(m,  n) : \Pi (\alpha : \type{m}) (\beta : \alpha \rightarrow \type{n}) (x : \alpha) (y : \beta\ x), \alpha
}\\\\
\end{equation}

\newpage

Note that this rule for both uses the point-free notation mentioned above. The judgment \(f : \beta\) is to denote \(\forall (x : \alpha), \beta(x)\). This is because our equivalent of \(\Pi\) types can be applied, unlike in the CoC.

\begin{equation}
  \frac{}{
    \text{both}(m, n, o) : \Pi (\alpha : \type{m}) (\beta : \alpha \rightarrow \type{n}) (\gamma : \alpha \rightarrow \type{o}) (f : \beta) (g : \gamma) (x : \alpha), (\beta(x) \times \gamma(x))
  }
\end{equation}

\begin{equation}
  \frac{}{
    \type{n} : \type{n + 1}
  }
\end{equation}

\begin{equation}
  \frac{}{
    
  }
\end{equation}

What's left?

\begin{itemize}
  \item{cons}
  \item{app}
  \item{eq, but if I don't get to it, that's fine}
\end{itemize}

\end{document}
