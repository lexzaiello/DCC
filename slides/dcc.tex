\documentclass{beamer}
\usepackage{fontspec}
\usepackage{minted}
\usepackage[
  backend=biber,
  style=alphabetic,
  sorting=ynt
]{biblatex}
\addbibresource{./refs.bib}

\newenvironment{bitemize}[1]{\begin{itemize}\setlength{\itemsep}{1em}#1\end{itemize}}

\AtBeginSection[]{
  \begin{frame}
    \vfill
    \centering
    \begin{beamercolorbox}[sep=8pt,center,shadow=true,rounded=true]{title}
      \usebeamerfont{title}\insertsectionhead\par%
    \end{beamercolorbox}
    \vfill
  \end{frame}
}

\setmainfont{DejaVu Serif} 
\setsansfont{DejaVu Sans}
\setmonofont{JuliaMono}[
    Scale=0.9,
    Extension = .ttf,
    UprightFont = *-Regular,
    BoldFont = *-Bold,
    ItalicFont = *-RegularItalic,
    BoldItalicFont = *-BoldItalic
]

\usemintedstyle{tango}

\newcommand{\leane}[1]{\mintinline{lean4}|#1|}
\newcommand{\pie}[1]{\Pi_{#1}}
\newcommand{\type}[1]{\text{Type } #1}
\newcommand{\turnstile}{\vdash}
\newcommand{\sigl}[1]{\(\texttt{::[}#1\texttt{]}\)}
\newcommand{\sigle}[1]{\texttt{::[}#1\texttt{]}}

\usetheme{Madrid}

\title{The Dependent Combinator Calculus}
\author{Alexandra Aiello}
\date{\today}

\begin{document}

\frame{\titlepage}

\begin{frame}{Problem Statement: The \(\Pi\)-Combinator Mismatch}
  \begin{itemize}
    \item Lambdas are curried \(\lambda x.\lambda y.M\), and so are their types \(\forall (x : \alpha), \forall (y : \beta), \ldots\).
    \item Combinators are uncurried (\(K(x, y) = x\). Fixed number of known arguments.
    \item Past approaches treated combinator types as uncurried.
      \begin{itemize}
        \item Altenkirch and I used \(\Pi\) combinators, or meta combinators. Many disadvantages.
      \end{itemize}
    \item By uncurrying combiantor types, dependent types become simple, and elegant.
  \end{itemize}    
\end{frame}

\begin{frame}[fragile]{Central Thesis: Combinator-Sigma Correspondence}
  \begin{figure}
\begin{minted}[mathescape, fontsize=\small, linenos]{lean4}
/-- Interpret a function on `Σ x : α, β x` as a dependent function with two arguments.
This also exists as an `Equiv` as `Equiv.piCurry γ`. -/
def Sigma.curry {γ : ∀ a, β a → Type*} (f : ∀ x : Sigma β, γ x.1 x.2) (x : α) (y : β x) : γ x y :=
  f ⟨x, y⟩

/-- Interpret a dependent function with two arguments as a function on `Σ x : α, β x`.
This also exists as an `Equiv` as `(Equiv.piCurry γ).symm`. -/
def Sigma.uncurry {γ : ∀ a, β a → Type*} (f : ∀ (x) (y : β x), γ x y) (x : Sigma β) : γ x.1 x.2 :=
  f x.1 x.2

-- in Mathlib Equiv module
def piCurry {α} {β : α → Type*} (γ : ∀ a, β a → Type*) :
    (∀ x : Σ i, β i, γ x.1 x.2) ≃ ∀ a b, γ a b where
  toFun := Sigma.curry
  invFun := Sigma.uncurry
  left_inv := Sigma.uncurry_curry
  right_inv := Sigma.curry_uncurry
\end{minted}
    \caption{Equivalence between dependent functions and sigmas in Mathlib (curry-uncurry). Sigmas carry the variable and substitution as data, whereas \(\forall\) substitution is not data.}
  \end{figure}
\end{frame}

\begin{frame}{Central Thesis: Combinator-Sigma Correspondence}
  A function on a sigma term corresponds to a dependent combinator equation. A sigma type corresponds to the uncurried type of a dependent combinator. E.g., the dependent \(K\) combinator:
  \begin{equation}
    K((\alpha : Type), (\beta : \alpha \rightarrow Type), (x : \alpha), (y : \beta x)) = y
  \end{equation}
  \begin{minted}[mathescape, fontsize=\small, linenos]{lean4}
  \end{minted}
\end{frame}

\begin{frame}{Central Thesis: Combinator-Sigma Correspondence}
  \begin{itemize}
    A sigma term \leane{Sigma.mk (x : α) (y : ∀ (x : α), β x)} corresponds to a dependent combinator equation \leane{C(x, y) = M}. A sigma type \leane{
  \end{itemize}    
\end{frame}

\begin{frame}{Recap: Assertions / Arguments Register Design}
  \begin{itemize}
    \item Check an application by applying a list of the arguments \leane{:: x (:: y (:: z nil))} (using old notation) to the type of the function \leane{f : t_f}
    \item \leane{t_f} is a series of instructions {\leane{next}, \leane{read}, \leane{apply}, \ldots} to manipulate the arguments and output the \textbf{expected types} in the same order as the arguments.
    \item Was able to express dependent types, but the list combinators {\leane{next}, \leane{read}, \leane{apply}, \ldots} were almost dynamically typed.
    \item Couldn't handle currying.
    \item Design was not elegant at all. Special cases all over the place.
  \end{itemize}
\end{frame}

\begin{frame}{Summary: Latest Version}
  \begin{itemize}
    \item The latest version combines the combinators I showed you last time we met, and the asserts / args design in a well-typed context.
    \item Key insight: sigma types in Lean \leane{(fst : α) × (snd : β fst)} coincide perfectly with application in a list-based calculus, and give the previous iterations a well-typed interpretation.
      \begin{itemize}
        \item We still have normal application \leane{(f x)}, but now, expressions can be grouped like data and applied at a later time.
      \end{itemize}
  \end{itemize}
\end{frame}

\section{AST}

\begin{frame}[fragile]{AST}
  \begin{minted}[mathescape, fontsize=\small, linenos]{lean4}
abbrev Level := ℕ

inductive Expr where
  | ty     : Level → Expr
   -- Cons can be partially applied, so it is a combinator
  | cons   : Expr
  | app    : Expr → Expr → Expr
   -- Nil "downgrades" a term to a type
   -- nil (α : Ty m) : α → Ty m
   -- useful for typing arguments like ∀ (α : Type), ...
  | nil    : Level → Expr
  -- The I combinator in SK
  | id     : Level → Expr
  -- Dependent K combinator
  | const  : Level → Level → Expr
  -- Nondependent K. Closes the loop of type dependency.
  | const' : Level → Level → Expr
\end{minted}

Cont \ldots
\end{frame}

\begin{frame}[fragile]{AST (cont \ldots)}
  \begin{minted}[mathescape, fontsize=\small, linenos]{lean4}
  inductive Expr where
  ...
  /- form a new sigma pair from (both f g x) ::[(f x), (g x)]
     but does not force application of (f x) (g x)
     combined with snd projection results in S -/
  | both   : Level → Level → Level → Expr
  -- Branch on def-eq. For practical applications.
  | eq     : Level → Level → Level → Expr
  \end{minted}
\end{frame}

\begin{frame}[fragile]{Evaluation Rules}
  \begin{minted}[mathescape, fontsize=\small, linenos]{lean4}
inductive is_step : Expr → Expr → Prop
  | sapp   : is_step ($ ::[x, f], fn) ($ fn, f, x)
  | nil    : is_step ($ (nil _o), α, _x) α
  | id     : is_step ($ (.id _o), _α, x) x
  | const  : is_step ($ (.const _o _p), _α, _β, c, _x) c
  | const' : is_step ($ (.const' _o _p), _α, _β, c, _x) c
  | both   : is_step ($ (.both _o _p _q), _α, _β, _γ, f, g, x)
    ::[($ f, x), ($ g, x)]
  | eq_yes : a == b → is_step
    ($ (.eq _o _p), _α, _β, fn_yes, fn_no, a, b)
    (.app fn_yes a)
  | eq_no  : a ≠ b → is_step
    ($ (.eq _o _p), _α, _β, fn_yes, fn_no, a, b)
    (.app fn_no b)
  | left   : is_step f f' → is_step ($ f, x) ($ f', x)
  | right  : is_step x x' → is_step ($ f, x) ($ f, x')
  \end{minted}
\end{frame}

\begin{frame}[fragile]{Inference Rule: Application}
  \begin{figure}
    \begin{columns}[c]
      \begin{column}{0.5\textwidth}
        \centering App in CoC:
        \[
          \frac{\text{$\Gamma \vdash M : (\forall x : A.B)$} \quad\quad \text{$\Gamma \vdash N : A$}}{\Gamma \vdash M N : B[x := N]}
        \]
    \end{column}
    \begin{column}{0.5\textwidth}
      \centering App in DCC:
      \[
        \frac{\text{$\Gamma \vdash M : \sigle{A, B}$} \quad\quad \text{$\Gamma \vdash N : A$}}{
          \Gamma \vdash M N : \sigle{N, A} B}
      \]
    \end{column}
  \end{columns}
  \end{figure}
\end{frame}

\end{document}
